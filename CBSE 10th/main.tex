\let\negmedspace\undefined
\let\negthickspace\undefined
\documentclass[journal,12pt,twocolumn]{IEEEtran}
\usepackage{gensymb}
\usepackage{amssymb}
\usepackage[cmex10]{amsmath}
\usepackage{amsthm}
\usepackage[export]{adjustbox}
\usepackage{bm}
\usepackage{longtable}
\usepackage{enumitem}
\usepackage{mathtools}
\usepackage[breaklinks=true]{hyperref}
\usepackage{listings}
\usepackage{color}                                            %%
\usepackage{array}                                            %%
\usepackage{longtable}                                        %%
\usepackage{calc}                                             %%
\usepackage{multirow}                                         %%
\usepackage{hhline}                                           %%
\usepackage{ifthen}                                           %%
\usepackage{lscape}     
\usepackage{multicol}
\usepackage{marvosym}
% \usepackage{enumerate}
\DeclareMathOperator*{\Res}{Res}
\renewcommand\thesection{\arabic{section}}
\renewcommand\thesubsection{\thesection.\arabic{subsection}}
\renewcommand\thesubsubsection{\thesubsection.\arabic{subsubsection}}
\renewcommand\thesectiondis{\arabic{section}}
\renewcommand\thesubsectiondis{\thesectiondis.\arabic{subsection}}
\renewcommand\thesubsubsectiondis{\thesubsectiondis.\arabic{subsubsection}}
\hyphenation{op-tical net-works semi-conduc-tor}
\def\inputGnumericTable{}                                 %%

\lstset{
frame=single, 
breaklines=true,
columns=fullflexible
}
\begin{document}
\newtheorem{theorem}{Theorem}[section]
\newtheorem{problem}{Problem}
\newtheorem{proposition}{Proposition}[section]
\newtheorem{lemma}{Lemma}[section]
\newtheorem{corollary}[theorem]{Corollary}
\newtheorem{example}{Example}[section]
\newtheorem{definition}[problem]{Definition}
\newcommand{\BEQA}{\begin{eqnarray}}
\newcommand{\EEQA}{\end{eqnarray}}
\newcommand{\define}{\stackrel{\triangle}{=}}
\bibliographystyle{IEEEtran}
\providecommand{\mbf}{\mathbf}
\providecommand{\pr}[1]{\ensuremath{\Pr\left(#1\right)}}
\providecommand{\qfunc}[1]{\ensuremath{Q\left(#1\right)}}
\providecommand{\sbrak}[1]{\ensuremath{{}\left[#1\right]}}
\providecommand{\lsbrak}[1]{\ensuremath{{}\left[#1\right.}}
\providecommand{\rsbrak}[1]{\ensuremath{{}\left.#1\right]}}
\providecommand{\brak}[1]{\ensuremath{\left(#1\right)}}
\providecommand{\lbrak}[1]{\ensuremath{\left(#1\right.}}
\providecommand{\rbrak}[1]{\ensuremath{\left.#1\right)}}
\providecommand{\cbrak}[1]{\ensuremath{\left\{#1\right\}}}
\providecommand{\lcbrak}[1]{\ensuremath{\left\{#1\right.}}
\providecommand{\rcbrak}[1]{\ensuremath{\left.#1\right\}}}
\theoremstyle{remark}
\newtheorem{rem}{Remark}
\newcommand{\sgn}{\mathop{\mathrm{sgn}}}
\providecommand{\abs}[1]{\left\vert#1\right\vert}
\providecommand{\res}[1]{\Res\displaylimits_{#1}} 
\providecommand{\norm}[1]{\left\lVert#1\right\rVert}
%\providecommand{\norm}[1]{\lVert#1\rVert}
\providecommand{\mtx}[1]{\mathbf{#1}}
\providecommand{\mean}[1]{E\left[ #1 \right]}
\providecommand{\fourier}{\overset{\mathcal{F}}{ \rightleftharpoons}}
%\providecommand{\hilbert}{\overset{\mathcal{H}}{ \rightleftharpoons}}
\providecommand{\system}{\overset{\mathcal{H}}{ \longleftrightarrow}}
	%\newcommand{\solution}[2]{\textbf{Solution:}{#1}}
\newcommand{\solution}{\noindent \textbf{Solution: }}
\newcommand{\cosec}{\,\text{cosec}\,}
\providecommand{\dec}[2]{\ensuremath{\overset{#1}{\underset{#2}{\gtrless}}}}
\newcommand{\myvec}[1]{\ensuremath{\begin{pmatrix}#1\end{pmatrix}}}
\newcommand{\mydet}[1]{\ensuremath{\begin{vmatrix}#1\end{vmatrix}}}
\numberwithin{equation}{subsection}
\makeatletter
\@addtoreset{figure}{problem}
\makeatother
\let\StandardTheFigure\thefigure
\let\vec\mathbf
\renewcommand{\thefigure}{\theproblem}
\def\putbox#1#2#3{\makebox[0in][l]{\makebox[#1][l]{}\raisebox{\baselineskip}[0in][0in]{\raisebox{#2}[0in][0in]{#3}}}}
     \def\rightbox#1{\makebox[0in][r]{#1}}
     \def\centbox#1{\makebox[0in]{#1}}
     \def\topbox#1{\raisebox{-\baselineskip}[0in][0in]{#1}}
     \def\midbox#1{\raisebox{-0.5\baselineskip}[0in][0in]{#1}}
\vspace{3cm}
\title{Maths 30/3, 2015}
\author{DHRUV PARASHAR}
\maketitle
\newpage
\bigskip
\renewcommand{\thefigure}{\theenumi}
\renewcommand{\thetable}{\theenumi}
\section{Section A}
\renewcommand{\theequation}{\theenumi}
\begin{enumerate}[label=\thesection.\arabic*.,ref=\thesection.\theenumi]
\numberwithin{equation}{enumi}
\item In Figure \ref{Figure 1}, a tower AB is $20$ m high and BC, its shadow on the ground,is $20\sqrt{3}$ m long. Find the Sun’s altitude.
\begin{figure}[h!]
	\centering
    \includegraphics[width=0.8\columnwidth,center]{./figs/image1.png}
	\caption{}
	\label{Figure 1}
\end{figure}
\item Two different dice are tossed together. Find the probability that the product of the two numbers on the top of the dice is $6$.
\item If the quadratic equation
\begin{align*} 
px^2 – 2 \sqrt{5} px + 15 = 0
\end{align*}
has two equal roots,then find the value of $p$.
\item In \ref{Figure 2}, PQ is a chord of a circle with centre O and PT is a tangent. If $\angle \text{QPT} = 60 \degree,$ Find $\angle \text{PRQ} $.
\begin{figure}[h!]
	\centering
    \includegraphics[width=0.8\columnwidth,center]{./figs/image2.png}
	\caption{}
	\label{Figure 2}
\end{figure}
\end{enumerate}
\section{Section B}
\renewcommand{\theequation}{\theenumi}
\begin{enumerate}[label=\thesection.\arabic*.,ref=\thesection.\theenumi]
\numberwithin{equation}{enumi}
\item In an AP, if 
\begin{align*}
 S_5 + S_7 &= 167\\  
 S_{10}&= 235
\end{align*}
then find the AP, where $S_n$ denotes the sum of its first $n$ terms.
\item The points 
\begin{align*}
 \vec{A}&= \myvec{4 \\ 7}\\  
 \vec{B}&=\myvec{p \\ 3}\\  
 \vec{C}&=\myvec{7 \\ 3}
\end{align*}
are the vertices of a right triangle, right-angled at $ \vec{B} $. Find the value of $p$.
\item In Figure \ref{Figure 3}, two tangents RQ and RP are drawn from an external point R to the circle with centre O. If $\angle \text{PRQ} = 120 \degree,$ then prove that OR = PR + RQ.
  \begin{figure}[h!]
	\centering
    \includegraphics[width=0.8\columnwidth,center]{./figs/image3.png}
	\caption{}
	\label{Figure 3}
\end{figure}
\item In Figure \ref{Figure 4}, a triangle ABC is drawn to circumscribe a circle of radius $3$ cm, such that the segments BD and DC are respectively of lengths $6$ cm  and $9$ cm. If the area of $\triangle \text{ABC is }54 cm^2$, then find the lengths of sides AB and AC.
\begin{figure}[h!]
	\centering
    \includegraphics[width=0.8\columnwidth,center]{./figs/image4.png}
	\caption{}
	\label{Figure 4}
\end{figure}
\item Find the relation between x and y if the points 
\begin{align*}
\vec{A}&=\myvec{x \\ y}\\ 
\vec{B}&=\myvec{-5\\7}\\
\vec{C}&=\myvec{-4\\5} 
\end{align*}
are collinear.
\item Solve the following quadratic equation for $x$:
 \begin{align*}
     4x^2 + 4bx – (a^2–b^2) = 0 
 \end{align*}
\end{enumerate}
\section{Section C}
\renewcommand{\theequation}{\theenumi}
\begin{enumerate}[label=\thesection.\arabic*.,ref=\thesection.\theenumi]
\numberwithin{equation}{enumi}
\item Due to sudden floods, some welfare associations jointly requested the government to get $100$ tents fixed immediately and offered to contribute $ 50\% $ of the cost. If the lower part of each tent is of the form of a cylinder of diameter $4.2$ m and height $4$ m with the conical upper part of same diameter but of height $2.8$ m, and the canvas to be used costs Rs. $100$ per sq. m, find the amount, the associations will have to pay. What values are shown by these associations ? [Use $\pi=\dfrac{22}{7}$]
\item A hemispherical bowl of internal diameter $36$ cm contains liquid. This liquid is filled into $72$ cylindrical bottles of diameter $6$ cm. Find the height of the each bottle, if $10 \% $ liquid is wasted in this transfer.
\item A cubical block of side $10$ cm is surmounted by a hemisphere. What is the largest diameter that the hemisphere can have ? Find the cost of painting the total surface area of the solid so formed, at the rate of Rs. $5$ per $100$ sq. cm. [Use $\pi= 3.14$]
\item $504$ cones, each of diameter $3.5$ cm and height $3$ cm, are melted and recast into a metallic sphere. Find the diameter of the sphere and hence find its surface area. [Use $\pi=\dfrac{22}{7}$]
\item Solve for x :
 \begin{align*}
     \sqrt{3}x^2 -2\sqrt{2}x-2\sqrt{3}= 0 
 \end{align*}
\item The angle of elevation of an aeroplane from a point A on the ground is $60 \degree  $. After a flight of $15$ seconds, the angle of elevation changes to $  30 \degree.$ If the aeroplane is flying at a constant height of $1500\sqrt{3}$ m, find the speed of the plane in km/hr.
\item Find the area of the minor segment of a circle of radius $14$ cm, when its central angle is $60 \degree.$ Also find the area of the corresponding major segment. [Use $\pi =\dfrac{22}{7}$]
\item The $13^{th}$ term of an AP is four times its $3^{rd}$ term. If its fifth term is $16$, then find the sum of its first ten terms.
\item Find the coordinates of a point P on the line segment joining
\begin{align*}
\vec{A} &= \myvec{1\\2}\\
\vec{B} &= \myvec{6\\7}
\end{align*}
such that AP = $\dfrac{2}{5}$AB.
\item A bag contains, white, black and red balls only. A ball is drawn at random from the bag. If the probability of getting a white balls is $\dfrac{3}{10}$ and that of black ball is $\dfrac{2}{5}$, then find the probability of getting a red ball. If the bag contains $20$ black balls, then find the total number of balls in the bag.
\end{enumerate}
\section{Section D}
\renewcommand{\theequation}{\theenumi}
\begin{enumerate}[label=\thesection.\arabic*.,ref=\thesection.\theenumi]
\numberwithin{equation}{enumi}
\item At a point A, $20$ metres above the level of water in a lake, the angle of elevation of a cloud is $30 \degree.$ The angle of depression of the reflection of the cloud in the lake, at A is $60\degree .$ Find the distance of the cloud from A.
\item A card is drawn at random from a well-shuffled deck of playing cards. Find the probability that the card drawn is
 \begin{enumerate}
     \item a card of spade or an ace.
     \item a black king. 
     \item neither a jack nor a king.
     \item either a king or a queen
 \end{enumerate}
\item In \ref{Figure 5}, PQRS is a square lawn with side PQ = $42$ metres. Two circular flower beds are there on the sides PS and QR with centre at O, the intersection of its diagonals. Find the total area of the two flower beds (shaded parts).
 \begin{figure}[h!]
	\centering
    \includegraphics[width=0.8\columnwidth,center]{./figs/image5.png}
	\caption{}
	\label{Figure 5}
\end{figure}
\item From each end of a solid metal cylinder, metal was scooped out in hemispherical form of same diameter. The height of the cylinder is $10$ cm and its base is of radius $4.2$ cm. The rest of the cylinder is melted and converted into a cylindrical wire of $1.4$ cm thickness. Find the length of the wire. [Use $ \pi=\frac{22}{7} $]
\item The diagonal of a rectangular field is $16$ metres more than the shorter side. If the longer side is $14$ metres more than the shorter side, then find the lengths of the sides of the field.
\item Prove that the lengths of the tangents drawn from an external point to a circle are equal.
\item Prove that the tangent drawn at the mid-point of an arc of a circle is parallel to the chord joining the end points of the arc.
\item A truck covers a distance of $150$ km at a certain average speed and then covers another $200$ km at an average speed which is $20$ km per hour more than the first speed. If the truck covers the total distance in $5$ hours, find the first speed of the truck.
\item An arithmetic progression $5, 12, 19, ...$ has $50$ terms. Find its last term. Hence find the sum of its last $15$ terms.
\item Construct a triangle ABC in which AB = $5$ cm, BC = $6$ cm and $\angle \text{ABC} = 60\degree$. Now construct another triangle whose sides are $\dfrac{5}{7}$ times the corresponding sides of $\triangle \text{ABC}$.
\item Find the values of k for which the points
\begin{align*}
\vec{A} &= \myvec{k+1\\2k}\\
\vec{B} &= \myvec{3k\\2k+3}\\
\vec{C} &= \myvec{5k-1\\5k}
\end{align*}
are collinear.
\end{enumerate}
\end{document}
